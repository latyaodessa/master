\section{Developer Guide}\label{Developer Guide}

\subsection{System configuration}\label{System configuration}

\subsubsection{Setting up the Monitoring System}\label{Monitoring System setup}

 In the global configuration is possible to setup scare interval and evaluation interval.
global:
 
 \begin{lstlisting}
  scrape_interval:     15s 
  evaluation_interval: 15s 
\end{lstlisting}

Prometheus has to reference on Alert Manager, where messages will be published. 
 \begin{lstlisting}
alerting:
  alertmanagers:
  - static_configs:
    - targets:
       - localhost:9093
\end{lstlisting}

Every machine where Prometheus is installed can has its own alerting rules. In general alerting rules are located in the root folder of Prometheus.

 \begin{lstlisting}
rule_files:
   - "alert.rules"
   - "node.rules"
   - "test.rules"
\end{lstlisting}

Since there is a need to get more specific data, in N2Sky was decided to user Node Exporter Module. The reference on this module has to be added into configuration

 \begin{lstlisting}
- job_name: 'node'
    scrape_interval: 5s
    target_groups:
-	targets: ['localhost:9100']
\end{lstlisting}

Node Exporter Module has no configuration file. Prometheus listen the modules and scrap the data with a defined interval.

For deploying alert manager Docker containers technology is used.
\hl{TODO}

\subsubsection{Setting up Alert Management System}\label{Setting up Alert Management System}

All configuration of alert manager are written in YAML file. 
On the beginning SMTP email sender should be configured. This would be used to sending notifications.

 \begin{lstlisting}
global:
  smtp_smarthost: 'localhost:25'
  smtp_from: 'alertmanager@example.org'
  smtp_auth_username: 'alertmanager'
  smtp_auth_password: 'password'
\end{lstlisting}

It is possible to define multiple Email templates and configure which template need to be loaded on which severe level. In configuration the path to templates need to be defined. 

 \begin{lstlisting}
templates: 
-	'/etc/alertmanager/template/*.tmpl'
\end{lstlisting}

When alerts are consumed they need to be converted using Email template and fired to the particular route. Every route has a receiver. 

 \begin{lstlisting}
route:
group_by: ['alertname', 'cluster', 'service']
group_wait: 30s
group_interval: 5m
repeat_interval: 3h 
receiver: team-X-mails
\end{lstlisting}

\begin{description}
\item[group\_by] Group by label. This way ensures that multiple alerts from difference cluster can be received
\item[group\_wait] Ensures that multiple alerts can be fired shortly after particular group is received.
\item[group\_interval] Interval between alert batches.
\item[Receiver] Unique name of receiver which is defined in configuration. 
\end{description}

Receiver it is a group of matching by regular expression events.

 \begin{lstlisting}
  routes:
- match_re:
      service: ^(foo1|foo2|baz)
    receiver: team-X-mails
\end{lstlisting}

Receiver can be defined by user configuration, it is an email where is alert notification will be send.

 \begin{lstlisting}
receivers:
- name: 'team-X-mails'
  email_configs:
  - to: 'team-X+alerts@example.org'
\end{lstlisting}

\paragraph{How to write alerting rules} 

The alerting rules are supporting simple query language, which looks very similar to Sequel Query Language.  
There is multiple possibilities how work with a alerting rules. The query language allows to use an expression and as a result to check an attribute of time series. 

 \begin{lstlisting}
ALERT HighLatency
 IF api_http_request_latencies_second{quantile="0.7"} > 1 
FOR 5m 
LABELS { severity="critical"} 
ANNOTATIONS {   summary = "High latency detected ",   description = "over limit? } 
\end{lstlisting}

Following notations should be considered be creation of alerting rules:
\begin{itemize}
\item All queries staring with "ALERT" namespace. After it follows name of alert in this case it is "HighLatency".  
\item "IF" is a condition "api\_http\_request\_latencies\_second", which based on Prometheus Tool expression.  Set of time series with this expression has one parameter it is "quantile". Reading condition as a whole can be translated in a human language like this: "Send a alert if latency request per second bigger then 0.7". 
\item "FOR" it is period of time how often this condition should be checked. 
\item "LABELS" shows a severity level. There are 3 types of severity:
\begin{itemize} 
        \item Critical
        \item Warning
        \item Page
     \end{itemize}
\item Every severity level can be defined on developer needs.
\item "ANNOTATIONS" shows a readable for human comments. There are two sub sections: summary, which shows a short description of the event and description where detailed information about deviation can be written
\end{itemize}

For deploying alert manager Docker containers technology is used.
\hl{TODO}

\subsection{Continuous integration}\label{Continuous integration}
\subsection{API Documentation}\label{API Documentation}
\subsubsection{N2Sky Monitoring System API Documentation}\label{N2Sky Monitoring System API Documentation}

