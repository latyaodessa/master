\section{Introduction}\label{introduction}

The master thesis has a tight connection with the infrastructure design for the N2Sky System proposed by Aliaksandr Adamenko \cite{adamenko}. The basis of the infrastructure is a microservices architecture. It is implemented and designed using concepts of decomposition, single responsibility and continuous integration, which allows to scale the system across the share-nothing infrastructure and gives more control over the computational resources. The architecture is designed to provide easier natural support for Cloud deployment with distributed computational resources.

\begin{description}
\item[Objectives:] Develop the ready-to-use N2Sky System, which can be easily constructed from scratch and be maintainable on purpose. Implement scalable application, which can be used on any device. Introduce ViNNSL template in order to use this specification language across the whole application. Apply microservices architecture on the backend as well as on the frontend side. Organize monitoring and alerting system on N2Sky cloud and its instances. Make the live demo-prototype for testing purposes. 
\item[Non-objectives:] The thesis contains information about the future production system. The current version is not fully ready and not tested in order to be released. 
\end{description}

\subsection{Motivation}\label{motivation}

Artificial neural networks have a long history. It is referred to cybernetics and connectionism.  After the introduction of "deep learning", scientists started to use the full potential of neural networks \cite{Goodfellow}. 

The director of AI (Artificial Intelligence) at Tesla, Andrej Karpathy, described the main reasons why the AI was held back:

\begin{itemize}
\item \emph{Compute} (the obvious one: Moore's Law, GPUs, ASICs)
\item \emph{Algorithms} (in a nice form, not just out there somewhere on the internet)
\item \emph{Infrastructure}  (software under you - Linux, TCP/IP, Git, ROS, PR2, AWS, AMT, TensorFlow, etc.) \cite{andrej}
\end{itemize}

These blockers have disappeared today, which gives engineers and scientists the possibility to investigate and create projects based on AI. 

Nowadays, there is a dozen of different types of neural networks and existing models. The need to have a platform to consolidate them all is crucial. The N2Sky platform gives the opportunity to simulate owned or already existing neural networks without the need of buying an expensive environment. The platform where neural network engineers can publish their own neural network and observe its behavior or where other users try to train it and evaluate the trained model. N2Sky is this kind of platform, which gives the possibility not only to use the N2Sky cloud for their needs but also to try out and study neural network field without deep knowledge in this field. 



\subsection{Terms and Definitions}\label{Terms and definitions}

\begin{description}
\item[N2Sky] is an artificial neural network simulation platform, which implements the Neural Network as a Service approach. It is a cloud-based system, which implements microservices architecture on infrastructure as well as on application level. The system was motived by delivering an intuitive tool for different stakeholders. The arbitrary user looking for existing neural network solutions, the neural network engineer creating a neural network based on existing neural network paradigms, the contributor implementing and publishing it on N2Sky repository. Since N2Sky is also a simulation platform, every stakeholder can perform training of the desired neural network as well as evaluate trained models.  N2Sky today provides a virtual collaboration platform to the computational intelligence community.
\item[ViNNSL template] is an adopted for N2Sky application ViNNSL language \cite{Beran2008} , which contains the neural network description, metadata, structure of the neural network and its connection. The ViNNSL template is used across the entire N2Sky application, on the frontend as a representation and on services side as a database model.
\item[Micro application] is an implementation of the microservices approach, which is normally used on the infrastructure side, but in this case it was used on the frontend application. The micro application is a small application, which is responsible for limited and consolidated workflows. The micro applications can be bundled with other applications and work as a whole so that the user has the feeling that he is using only one application.
\end{description}



\subsection{Related Work}\label{Related work}

The basis for development and implementation of the N2Sky platform were various theoretical concepts which were reviewed and improved by multiple researchers.

The UK e-Science initiative~\cite{UKeS} describes several goals to be reached by fostering new stimulating techniques:

\begin{itemize}
  \item Enabling more effective and seamless collaboration of dispersed communities, both scientific and commercial.
  \item enable large-scale applications comprising of thousands of computers, large-scale pipelines etc.
  \item Transparent access to "high-end" resources from the desktop.
  \item Provide a uniform "look \& feel" to a wide range of resources
  \item Location independence of computational resources as well as data. 
\end{itemize}

Since large numbers of neural networks were developed, especially in the past few years, authors were motivated to create "everything about sharing" approach, which was implemented in N2Grid~\cite{schikuta2004n2grid} and N2Cloud~\cite{huqqani2010n2cloud}.

Unfortunately, both systems required deep knowledge in the neural network field and were almost not usable by the arbitrary users. The limited number of simulation and non-friendly user interface does not attract users, which leads to these common problems:

\begin{itemize}
\item \emph{Complexity.} The tool was too complex and did not have an intuitive user interface. 
\item \emph{Proprietary system} missing interconnection and data exchange to other software systems.
\item \emph{Lack of resources} for routine workflows.
\end{itemize}

The most related tool, which represents an expert system, was the CIML (Computational Intelligence and Machine Learning) ~\cite{zurada2009building}. The main idea of the tool was to create a web-based platform for machine learning, neural networks, and artificial intelligence, where any user could share and obtain resources. The project failed because its target was a too big diverse community and offered the all-in-one solution. Due to the complexity and lack of guidance, it failed. 

After this research, the original N2Sky \cite{N2Sky} was released. It was a cloud-based solution oriented only on neural networks. The idea was that any user could run the simulation on any device. The system supported multiple stakeholders with different roles and it also had a mobile web portal. 

The original N2Sky supported ViNNSL language. ViNNSL is XML-based domain specific language, which was developed as a communication framework to support the service-oriented architecture based on neural network environments. 

The decision was to take the good parts, refactor them, apply new and trending technologies in order to approach the production-ready system. In this master thesis, the novel N2Sky system was developed just by using the concepts, trial and failures from previous generations. 

